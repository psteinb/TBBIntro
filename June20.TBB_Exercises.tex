\documentclass[a4paper]{article}
\usepackage[latin1]{inputenc}
\usepackage[english]{babel}
\usepackage{a4wide}
%\usepackage{hyperref}
\usepackage{graphics}
\usepackage{color}
\usepackage{listings}
\lstset{%
  language=C++,
  numbers=left, 
  numberstyle=\tiny,
  frame=single
}
\renewcommand{\theenumi}{\alph{enumi}}
\renewcommand{\labelenumi}{\theenumi)}

\begin{document}
\part*{Pragmatic Introduction to Intel's Threading Building Blocks}
\label{part:all}

 
\section*{Example 1}
\large
A simple cout of a for loop!\\
  \tiny
   \begin{lstlisting}[float=ht,caption={\large \textbf{Serial}}]
for(int i = 0;i<nMax;i++){
  std::cout << "iteration "<< i << std::endl;
}
   \end{lstlisting}
\\
   \begin{lstlisting}[float=ht,caption={\large \textbf{Parallel}}]
struct NumPrinter
{
   void operator()(const tbb::blocked_range<size_t>& r) const {
    tbb::blocked_range<size_t>::const_iterator rangeItr = r.begin();
    tbb::blocked_range<size_t>::const_iterator rangeEnd = r.end();
   
    for (;rangeItr!=rangeEnd; ++rangeItr)
    {
      std::cout << "iteration "<< rangeItr << std::endl;
    }
  }
};

int main(int argc, char* argv[])
{

  //...
  int grainsize = nIterations/nThreads;
  tbb::parallel_for(tbb::blocked_range<size_t>(0,
                                               nIterations,
                                               grainsize),
                    NumPrinter()
                    );
  //...
}
   \end{lstlisting}
  
% \section*{Exercise 2}
% Imagine we are writing a small piece of software to draw various geometric objects.\\
% \small
% \lstinputlisting[caption={\large \textbf{The Square/Circle Problem}},label=SqrCirc]{code/OCP.cc} 
% \normalsize
% \begin{enumerate}
% \item How many responsibilities has \texttt{DrawAllShapes} in Listing \ref{SqrCirc}?
% \item We are adding a new class \texttt{Triangle} and we want it to be drawn as well. How does Listing \ref{SqrCirc} adapt to this?
% \end{enumerate}

\end{document}
 