\documentclass[9pt,dvips]{beamer}
\usepackage[latin1]{inputenc}
\usepackage[english]{babel}
\usepackage{graphics}
%\usepackage{includegraphix}
\usepackage{hyperref}
\usepackage{units}

\usepackage{amsmath}
\usepackage{amsfonts}
\usepackage{amssymb}

\usepackage{color}

\usepackage{amsmath}
\usepackage{tikz}
\usetikzlibrary{calc}
\usetikzlibrary{shapes}

\useoutertheme{infolines}
\usetheme[]{default}
\setbeamertemplate{navigation symbols}{}
\setbeamertemplate{bibliography item}[text]
\setbeamertemplate{blocks}[rounded][shadow=true]
\setbeamertemplate{caption}[numbered]
\definecolor{somegrey}{RGB}{240,240,240}

\setbeamercolor{block title}{bg=black!80!white,fg=white}
\setbeamercolor{block title alerted}{use=alerted text,fg=blue!40!white,bg=black!80!white}

%\addfootbox{bg=grey,fg=black}{\quad \insertshortauthor \hspace{.4\textwidth} \insertshortinstitute \hspace{.4\textwidth} \insertpagenumber}

\begin{document}


\pgfdeclareimage[height=1.5cm]{ATLASlogo}{img/AN_atlaslogo}
\pgfdeclareimage[height=1.5cm]{FSPlogo}{img/FSPAtlas_logo}
\pgfdeclareimage[height=.8cm]{TUDlogo}{img/logo_schwarz}
\titlegraphic{ \hspace{2cm} \pgfuseimage{TUDlogo} \hfill \pgfuseimage{ATLASlogo} \hfill \pgfuseimage{FSPlogo}\hspace{2cm} }

\title[Class Design Principles]{ Class Design Principles in Object-Oriented Programming }
%\subtitle{- A  -}
\author[P. Steinbach]{Wolfgang F. Mader, \textbf{Peter Steinbach}}
\date{March 10th, 2011}

\institute[IKTP]{Institute for Nuclear and Particle Physics, TU Dresden}



\begin{frame}
  \vfill
  \begin{center}
    \begin{block}{From the Gaudi User Guide, \cite{gug}}
      \begin{quotation}
        A priori, we see no reason why moving to a language which supports the idea of objects, such as \texttt{C++}, should change the way we think of doing physics analysis.
      \end{quotation}
    \end{block}
    \huge
    Why Use Object-Oriented Programming in the First Place?
  \end{center}

  \vfill
\end{frame}

\maketitle

\begin{frame}
\frametitle{Outline}
\tableofcontents
\end{frame}

\section[Why OOP?]{Why Object-Oriented Programming?}
\subsection[Procedural]{Procedural Programming}
\begin{frame}
\frametitle{Procedural Programming}

\end{frame}

\subsection[OOP]{OOP Programming}
\begin{frame}
\frametitle{OO Programming}

\end{frame}

\subsection[OOP in HEP]{Programming Paradigms in HEP}
\begin{frame}
\frametitle{Programming Paradigms in HEP}

\end{frame}

\section{Orthogonality}

\section{Princinple 2}

\section{Princinple 3}

\section{Princinple 4}

\section{Summary}

\section{References}
\scriptsize
\begin{frame}
  \frametitle{References}
  \bibliographystyle{plain} 
  \bibliography{Mar10.ClassDesign}
\end{frame}
\end{document}




